%%%%%%%%%%%%%%%%%%%%%%%%%%%%%%%%%%%%%%%%%
% Short Sectioned Assignment
% LaTeX Template
% Version 1.0 (5/5/12)
%
% This template has been downloaded from:
% http://www.LaTeXTemplates.com
%
% Original author:
% Frits Wenneker (http://www.howtotex.com)
%
% License:
% CC BY-NC-SA 3.0 (http://creativecommons.org/licenses/by-nc-sa/3.0/)
%
%%%%%%%%%%%%%%%%%%%%%%%%%%%%%%%%%%%%%%%%%

%----------------------------------------------------------------------------------------
%	PACKAGES AND OTHER DOCUMENT CONFIGURATIONS
%----------------------------------------------------------------------------------------

\documentclass[letterpaper, fontsize=11pt]{scrartcl} % 8.5 x 11 paper and 11pt font size

\usepackage[T1]{fontenc} % Use 8-bit encoding that has 256 glyphs
\usepackage{fourier} % Use the Adobe Utopia font for the document - comment this line to return to the LaTeX default
\usepackage[english]{babel} % English language/hyphenation
\usepackage{amsmath,amsfonts,amsthm} % Math packages
\usepackage{graphicx}
\usepackage{caption}
\usepackage{wrapfig}
\usepackage{float}
\usepackage{amsmath}
\usepackage[english]{babel}
\usepackage{siunitx}

\usepackage{lipsum} % Used for inserting dummy 'Lorem ipsum' text into the template

\usepackage{sectsty} % Allows customizing section commands
\allsectionsfont{\centering \normalfont\scshape} % Make all sections centered, the default font and small caps

\usepackage{fancyhdr} % Custom headers and footers
\pagestyle{fancyplain} % Makes all pages in the document conform to the custom headers and footers
\fancyhead{} % No page header - if you want one, create it in the same way as the footers below
\fancyfoot[L]{} % Empty left footer
\fancyfoot[C]{} % Empty center footer
\fancyfoot[R]{\thepage} % Page numbering for right footer
\renewcommand{\headrulewidth}{0pt} % Remove header underlines
\renewcommand{\footrulewidth}{0pt} % Remove footer underlines
\setlength{\headheight}{13.6pt} % Customize the height of the header

\numberwithin{equation}{section} % Number equations within sections (i.e. 1.1, 1.2, 2.1, 2.2 instead of 1, 2, 3, 4)
\numberwithin{figure}{section} % Number figures within sections (i.e. 1.1, 1.2, 2.1, 2.2 instead of 1, 2, 3, 4)
\numberwithin{table}{section} % Number tables within sections (i.e. 1.1, 1.2, 2.1, 2.2 instead of 1, 2, 3, 4)

\setlength\parindent{0pt} % Removes all indentation from paragraphs - comment this line for an assignment with lots of text

\usepackage{mathtools}

\DeclarePairedDelimiter\abs{\lvert}{\rvert}%
\DeclarePairedDelimiter\norm{\lVert}{\rVert}%

% Swap the definition of \abs* and \norm*, so that \abs
% and \norm resizes the size of the brackets, and the 
% starred version does not.
\makeatletter
\let\oldabs\abs
\def\abs{\@ifstar{\oldabs}{\oldabs*}}
%
\let\oldnorm\norm
\def\norm{\@ifstar{\oldnorm}{\oldnorm*}}
\makeatother

%----------------------------------------------------------------------------------------
%	TITLE SECTION
%----------------------------------------------------------------------------------------

\newcommand{\horrule}[1]{\rule{\linewidth}{#1}} % Create horizontal rule command with 1 argument of height

\title{	
\normalfont \normalsize 
\textsc{Torque, Moment of Inertia, and Energy Conservation} \\ [25pt] % Your university, school and/or department name(s)
\horrule{0.5pt} \\[0.4cm] % Thin top horizontal rule
\huge Physics Lab 8 \\ % The assignment title
\horrule{2pt} \\[0.5cm] % Thick bottom horizontal rule
}

\author{Eric Welch} % Your name

\date{\normalsize{November 30, 2017}} % Today's date or a custom date

\begin{document}

\maketitle % Print the title

%----------------------------------------------------------------------------------------
%	PROBLEM 1
%----------------------------------------------------------------------------------------

\section{Hypothesis}

We will test the formula for rotational inertia $I$:

$$I=\sum m_i {r_i}^2$$

Where $r_i$ is the distance between axis of rotation and the center of mass of the individual masses $m_i$. Applying this to our apparatus, we can separate the sum into a fixed and changing term

\begin{align}
\begin{split}
I_T&=I_0+I\\
&=I_0+2Mr^2
\end{split}
\end{align}

Thus, a graph of dynamically measured $I_T$ vs. statically measured $I_0 +2Mr^2$ should be a straight line with a slope of 1.

\section{Error Analysis}

%------------------------------------------------

\subsection{X-Axis}

In this section, we calculate the maximum allowed error along the x-axis.\\

$x=I_0 + 2Mr^2$

\begin{align} 
\begin{split}
\Delta{x}\;=\;&\abs{\displaystyle{\frac{\partial{x}}{\partial{I_0}}}\Delta{r}}+\abs{\displaystyle{\frac{\partial{x}}{\partial{M}}}\Delta{M}}+\abs{\displaystyle{\frac{\partial{x}}{\partial{r}}}\Delta{r}}\\\\
=&\;\abs{\Delta{I_0}}+\abs{{r^2}\Delta{M}}+{\abs{2Mr\Delta{r}}}
\end{split}					
\end{align}

\textbf{Deltas:}\\
$\Delta{I_0}=\pm{0.000008}\; kg\cdot m^2$\\
$\Delta{M}=\pm{0.00005}\; kg$\\
$\Delta{r}=\pm{0.0025}\; m$\\

Thus, running our experimental values through the equation for $\Delta{x}$, we can assign error bars to each individual data point. Below is the table of $x$-axis error:\\

\begin{center}
    \begin{tabular}{ | l | p{3.7 cm} |}
    \hline
    Trial & Max x-Error ($kg\cdot m^2$) \\ \hline
    1 & 0.00001831 \\ \hline
    2 & 0.00001972 \\ \hline
    3 & 0.00001781 \\ \hline
    4 & 0.000006855 \\ \hline
    5 & 0.00001286 \\ \hline
    \end{tabular}
\end{center}

\pagebreak

%------------------------------------------------

\subsection{Y-Axis}

In this section, we calculate the maximum allowed error along the y-axis.\\

\begin{align}
\begin{split}
y=&I_T\\
=&m_Hr^2\Bigg(\frac{g-a}{a}\Bigg)
\end{split}
\end{align}

\begin{align}
\begin{split}
\Delta{y}\;=\;&\abs{\displaystyle{\frac{\partial{y}}{\partial{m_H}}}\Delta{m_H}}+\abs{\displaystyle{\frac{\partial{y}}{\partial{R}}}\Delta{R}}+\abs{\displaystyle{\frac{\partial{y}}{\partial{g}}}\Delta{g}}+\abs{\displaystyle{\frac{\partial{y}}{\partial{a}}}\Delta{a}}\\\\
=\;&\abs{R^2\Bigg(\frac{g-a}{a}\Bigg)\Delta{m_H}}+\abs{2m_HR\Bigg(\frac{g-a}{a}\Bigg)\Delta{R}}+\abs{m_HR^2\Bigg(\frac{1-a}{a}\Bigg)\Delta{g}}+\abs{\frac{-m_HR^2g}{a^2}\Delta{a}}
\end{split}
\end{align}

\textbf{Deltas:}\\
$\Delta{m_H}=\pm{0.00005}$ kg\\
$\Delta{R}=\pm{0.0002} m$\\
$\Delta{g}=\pm{0.01} \frac{m}{s^2}$\\
$\Delta{a}=\pm{0.5} \frac{m}{s^2}$\\

Thus, running our experimental values through the equation for $\Delta{y}$, we can assign error bars to each individual data point. Below is the table of $y$-axis error:\\

\begin{center}
    \begin{tabular}{ | l | p{4 cm} |}
    \hline
    Trial & Max y-Error ($kg\cdot m^2$) \\ \hline
    1 & 0.00748 \\ \hline
    2 & 0.03101 \\ \hline
    3 & 0.02075 \\ \hline
    4 & 0.004464 \\ \hline
    5 & 0.005214 \\ \hline
    \end{tabular}
\end{center}
\pagebreak
%----------------------------------------------------------------------------------------

\section{Data}
\subsection{Experimental Data}
{\footnotesize
\renewcommand{\arraystretch}{2}
\begin{center}
    \begin{tabular}{ | l | l | l | l | l | l | l | l | }
    \hline
   \textbf{Trial} & Acceleration \bigg($\frac{m}{s^2}$\bigg) & $M$ on Each Side of Post ($kg$) & $r$ to Mass ($m$) & $I_{statics} \;\bigg(\frac{kg \cdot m}{s}\bigg)$ & $I_{dynamics} \;\bigg(\frac{kg \cdot m}{s}\bigg)$ \\ \hline
    1 & 0.113 & 0.01225 & 0.150 & 0.000655 & 0.00164 \\ \hline
    2 & 0.0551 & 0.02225 & 0.165 & 0.00132 & 0.00337 \\ \hline
    3 & 0.0674 & 0.02225 & 0.150 & 0.0010 & 0.00275 \\ \hline
    4 & 0.146 & 0.02225 & 0.060 & 0.000264 & 0.00126 \\ \hline
    5 & 0.135 & 0.04225 & 0.060 & 0.000408 & 0.00137 \\ \hline
    \end{tabular}
\end{center}
}
\vspace{5mm}
\subsection{Other Data}
\vspace{4mm}
{\footnotesize
\renewcommand{\arraystretch}{2}
    \begin{tabular}{ | l | p{4 cm} |}
    \hline
    Mass on hangar - $m_H$ (kg) & 0.451 \\ \hline
    Radius of vertical post - $R$ ($m$) & 0.0065 \\ \hline
    Moment of inertia of vertical post - $I_0$ ($kg \cdot m^2$) & 0.000104 \\ \hline
    Error in mass measurements - $\Delta{m}$ (kg) & $\pm{0.01}$ \\ \hline
    Error in radius of vertical post - $\Delta{R}$ (m) & $\pm{0.0002}$ \\ \hline
    Error in acceleration - $\Delta{a}$ $\bigg(\frac{m}{s^2}\bigg)$ & $\pm{0.01}$ \\ \hline
    Error in moment of inertia of vertical post - $\Delta{I_0}$ ($kg \cdot m^2$) & $\pm{0.000008}$ \\ \hline
    Error in distance to $M$ - $\Delta{r}$ ($m$) & $\pm{0.0025}$ \\ \hline
    Error in gravitational field - $\Delta{g} \;\bigg(\frac{m}{s^2}\bigg)$ & $\pm{0.01}$ \\ \hline
    \end{tabular}
}
\vspace{5mm}
\subsection{Graphical Representation of Data}
See the next page for a graph of the relevant data.
\pagebreak
\section{Conclusion}

We were able to fit a straight line through all the data points within our allowed error ranges. Furthermore, our slope range includes a slope of 1 within its range.\\

$Slope_{max} = 32.7$\\

$Slope_{min}= -29.0$\\

$Slope_{best}=1.93$\\

Since data shows a linear relationship and since $Slope_{min}=-29.0< 1 < 32.7 = Slope_{max}$, our data supports the hypothesis.\\

Our $Slope_{best}$ of $1.93$ was a little off from the hypothetical outcome of $1$, but since our error ranges were so wide to begin with, the possible slope spectrum varied widely on either side of that.\\

Overall, this was a very hard experiment to measure precisely - the error ranges provide evidence of this.

\end{document}
